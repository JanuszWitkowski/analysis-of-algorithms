\documentclass{article} 

% NIE UŻYWAĆ BABELA!!
\usepackage[utf8]{inputenc}
\usepackage[T1]{fontenc}
\usepackage[MeX]{polski}
\usepackage{mathtools}
% \usepackage{amssymb}
\usepackage{amsmath,amssymb}
\usepackage{hyperref}
% \parindent=0pt % disables indentation
% \setlength{\hoffset}{-7.4mm}
\setlength{\oddsidemargin}{10mm}
\setlength{\evensidemargin}{4mm}
\setlength{\textwidth}{140mm}

\usepackage[ruled,vlined,linesnumbered,longend]{algorithm2e}
\newenvironment{pseudokod}[1][htb]{
	\renewcommand{\algorithmcfname}{}
	\begin{algorithm}[#1]%
	}{
\end{algorithm}
}
\DeclareMathOperator{\EX}{\mathbb{E}}% expected value
\DeclareMathOperator{\Var}{\mathbb{V}ar}% variance
\DeclareMathOperator{\sumk}{\sum_{k \ge 0}}

\title{Analiza Algorytmów - Zadanie 24}
\author{Janusz Witkowski 254663}
\date{21 kwietnia 2023}
\begin{document}
    \maketitle
    
    % \textbf{Hello World!} Today I am learning \LaTeX. \LaTeX{} is a great program for writing math. I can write in line math such as $a^2+b^2=c^2$. I can also give equations their own space: \[ \gamma^2+\theta^2=\omega^2\]

    \section{Zadanie 24}
    \subsection{Treść}
    Niech $X$ i $Y$ będą niezależnymi zmiennymi losowymi o funkcjach gęstości odpowiednio $f_X(x)$ oraz $f_Y(y)$. Dla $Z = X + Y$ pokaż, że

    \[ f_Z(z) = \int\displaylimits_{-\infty}^{\infty} f_X(x)f_Y(z - x) \,dx \]

    Jaki związek ma to zadanie z następnym zadaniem?

    \textit{Wskazówka:} zobacz \href{https://en.wikipedia.org/wiki/Convolution_of_probability_distributions}{splot rozkładów prawdopodobieństwa}. 


    \subsection{Rozwiązanie}
    \subsubsection{Splot rozkładów}

    Splot/sumę rozkładów prawdopodobieństwa definiuje się w teorii prawdopodobieństwa i statystyce jako operację w zakresie rozkładów prawdopodobieństwa, 
    która odpowiada dodawaniu niezależnych zmiennych losowych, a co za tym idzie, tworzeniu liniowych kombinacji zmiennych losowych. 
    Operacja ta jest szczególnym przypadkiem splotu w kontekście rozkładów prawdopodobieństwa.

    Rozkład prawdopodobieństwa sumy dwóch lub więcej niezależnych zmiennych losowych jest splotem ich indywidualnych rozkładów. 
    Jest to tłumaczone faktem, że funkcja gęstości prawdopodobieństwa (funkcja masy prawdopodobieństwa w przypadku dyskretnym) sumy niezależnych zmiennych losowych 
    jest splotem odpowiadających im funkcji gęstości prawdopodobieństwa (funkcji mas prawdopodobieństwa w przypadku dyskretnym). 
    % Wiele dobrze znanych rozkładów ma proste sploty.

    \subsubsection{Wyprowadzenie wzoru}

    Ze strony na Wikipedii o splocie rozkładów prawdopodobieństwa (dołączonej do wskazówki w treści zadania) możemy wyciągnąć ogólny wzór na rozkład złożonej z dwóch
    dyskretnych zmiennych losowych $X$ i $Y$ zmiennej losowej $Z = X + Y$:

    \begin{equation}
        P(Z=z) = P(X+Y=z) = P(X=k \land Y=z-k , k\in \mathbb{Z}) = \sum_{k=-\infty}^{\infty}{P(X=k \land Y=z-k)}
    \end{equation}

    Jeśli $X$ i $Y$ są niezależne, to

    \begin{equation}
        P(Z=z) = \sum_{k=-\infty}^{\infty}{P(X=k \land Y=z-k)} = \sum_{k=-\infty}^{\infty}{P(X=k)P(Y=z-k)}
    \end{equation}

    Przyjrzyjmy się teraz ciągłym rozkładom. Dla dowolnych ciągłych zmiennych losowych $X$ i $Y$ oraz $Z = X + Y$ mamy

    \begin{equation}
        f_Z(z) = \int\displaylimits_{-\infty}^{\infty} f_{XY}(x,z-x) \,dx 
    \end{equation}

    gdzie $f_{XY}$ jest \textit{funkcją wspólnej gęstości (joint probability density function)}, a przez $f_{XY}(x,z-x)$ rozumieć będziemy
    gęstość prawdopodobieństwa tego że $X=x \land Y=z-x$ dla danego $x\in \mathbb{R}$. 

    W zadaniu zakładamy, że $X$ i $Y$ są niezależne, z czego wynika (jak w przypadku dyskretnym)

    \begin{equation}
        f_{XY}(x,y)=f_X(x)f_Y(y)
    \end{equation}

    Stąd wychodzi nam oczekiwany wzór

    \begin{equation}
        f_Z(z) = \int\displaylimits_{-\infty}^{\infty} f_X(x)f_Y(z-x) \,dx 
    \end{equation}


    \subsubsection{Powiązanie z Zadaniem 25}

    W zadaniu 25 można wykorzystać pokazaną wyżej własność w dowodzie indukcyjnym. Idea jest taka, by w danym kroku indukcyjnym 
    policzyć gęstość za pomocą splotu rozkładu zmiennej losowej składającej się z sumy poprzednich zmiennych losowych oraz z następnej zmiennej losowej.
    Dzięki temu jesteśmy w stanie w prosty sposób przedstawić wzór na gęstość sumy zmiennych losowych o danych rozkładach. 

    \end{document}
